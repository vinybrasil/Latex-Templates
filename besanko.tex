%===========================================NAO MEXER DAQUI PRA BAIXO=========================================================================================
\documentclass[oneside,12pt, letterpaper]{book}
\usepackage{amsmath}
\usepackage{amssymb}
\usepackage{amsfonts} 
\usepackage{geometry} 
\usepackage{setspace}
\usepackage{helvet}
\usepackage{indentfirst}
\usepackage{graphicx}
\usepackage{subfig}
\graphicspath{ {./images/}}
%\renewcommand{\familydefault}{\sfdefault}
%duas linhas pra colocar no arial
\title{Teoria do equil{\'i}brio geral: resumo do cap{\'i}tulo 16 do livro do Besanko}
\date{\today}
\author {Vin{\' i}cyus A. Brasil}
% \geometry{textwidth=6in, textheight=9in, top=30mm, bottom=25mm}
\pagestyle{myheadings}
\usepackage[T1]{fontenc}
\usepackage{ae}
\usepackage[ansinew]{inputenc}
\usepackage[brazilian]{babel}
% \usepackage{graphicx}
\linespread{1.5} 
% horizontal
\setlength{\hoffset}{-1in}
\setlength{\oddsidemargin}{3.0cm} 
\setlength{\textwidth}{160mm} % (210mm - 30mm - 20mm)
\setlength{\parindent}{1.25cm} % identa��o de cada par�grafo
% vertical
\setlength{\voffset}{-1in}
\addtolength{\voffset}{2.0cm}
\setlength{\topmargin}{0.0cm}

\setlength{\headheight}{5mm}
\setlength{\headsep}{5mm}
\setlength{\textheight}{210mm} % (297mm - 30mm - 20mm)

%===========================================NAO MEXER DAQUI PRA CIMA=========================================================================================

\begin{document}

\maketitle
\date{}

\chapter{Defini{\c c}{\~a}o} 
\section{Equil{\'i}brio geral x equil{\'i}brio parcial}
A teoria do equil{\'i}brio parcial, normalmente abordada nas aulas de microeconomia, trata-se, em linhas gerais, do estudo da determina{\c c}{\~a}o de um 
pre{\c c}o espec{\'i}fico com todos os outros pre{\c c}os constantes. A teoria do equil{\'i}brio geral, busca, por sua vez, entender a determina{\c c}{\~a}o de v{\'a}rios pre{\c c}os 
simultaneamente. 

\section{Exemplo num{\'e}rico}
Suponha uma economia com dois produtos: caf{'e} e ch{'a}. Os dois produtos s{~a}o substitutos, ou seja, quando o pre{o de um bem aumenta, o pre{o do outro tamb{'e}m vai
aumentar, devido ao aumento no quantidade demandada. Suponha as seguintes equa{\c c}{~o}es de oferta e demanda para os dois bens, sendo C o caf{\'e} e T o ch{\'a}: 
\newline $ \begin{cases} Q^d_C = 120 - 50P_C + 40P_T \\ Q^d_T = 80 - 75P_T + 20P_C \\ Q^s_C = 80 + 20P_C \\ Q^s_T = 45 + 10P_T  \end{cases}   $

\chapter{Modelo simples de uma economia}
Suponha dois tipos de fam{\'i}lias: a de trabalhadores de escrit{'o}rio e a de oper{'a}rios. Cada tipo de fam{\'i}lia consome dois bens: energia e alimentos, sendo que
cada um deles {\'e} produzido com dois insumos: trabalho e capital. As fam{\'i}lias demandam energia e alimentos e ofertam trabalho e capital(ao alugarem
terras {\''a}s empresas, por exemplo. As empresas demandam capital e trabalho e ofertam energia e alimentos. Uma ilustra{{~a}o pode clarificar a ideia:
%FIGURA1
\section{Demanda das fam{\'i}lias: maximiza{\c c}{\~a}o de sua utilidade} 
Denotando a quantidade de energia consumida por $x$,a quantidade de alimentos consumida por $y$, as fam{\'i}lias de trabalhadores de escrit{\'o}rio por $W$ e dos
oper{\'a}rios por $B$. A utilidade delas {\'e} denotada por $U_W(x,y)$ e $U_B(x,y)$ e estamos supondo que as fam{\'i}lias de oper{\'a}rios sejam as principais ofertantes de
trabalho e as de escrit{\'o}rio sejam as principais ofertantes de capital. Nesta economia, h{\'a} mais oferta de trabalho que de capital, ou seja, a oferta agregada de
trabalho {\'e} maior que a oferta agregada de capital. O pre{\c c}o recebido por uma unidade de trabalho {\'e} $w$ e e o pre{\c c}o recebido por uma unidade de capital {\'e} $r$,
tendo que a renda das fam{\'i}lias, $I_B$ e $I_W$, depende dos dois fatores. Sendo o pre{\c c}o da unidade de energia $P_x$ e o pre{\c c}o de uma unidade de alimento $P_y$, cada
fam{\'i}lia toma os pre{\c c}os como dados e enfrenta o seguinte problema de maximiza{\c c}{\~a}o: \[ \text{max }U^W_{x,y}  \text{, sujeito {\`a}: }P_xx + P_yy = I_W(w,r)  \]
\[ \text{max }U^B_{x,y}  \text{, sujeito {\`a}: }P_xx + P_yy = I_B(w,r)  \] 
Tais fam{\'i}lias v{\~a}o maximizar a utilidade quando: 
\[ TMS^W_{x,y} = \frac{P_x}{P_y} \text{   e   } TMS^B_{x,y} = \frac{P_x}{P_y} \]
Pode-se, tamb{\'e}m, somar horizontalmente as demandas individuais de cada fam{\'i}lia e gerar a fun{\c c}{\~a}o de demanda agregada, como mostra a figura. Uma mudan{\c c}a no pre{\c c}o
do bem y deslocaria tais fun{\c c}{\~o}es. 
%FIGURA2
\section{Demanda das empresas: minimiza{\c c}{\~a}o de seus custos}
Nesta economia, algumas empresas produzem energia e outras produzem alimentos em mercados com concorr{\^e}ncia perfeita. Os retornos de escala s{\~a}o constantes (ao
duplicar os insumos, se duplica a produ{\c c}{\~a}o) e cada produtor individual de energia tem uma fun{\c c}{\~a}o  de produ{\c c}{\~a}o $x = f(l, k)$ e os de alimento $y = g(l, k)$, sendo $l$ e $k$ as quantidades individuais	
que cada produtor usa e $L$ e $K$ as quantidades agregadas de capital e trabalho no mercado. Logo, os produtores tem o seguinte problema de minimiza{\c c}{\~a}o:
\[ \text{min }wl+rk  \text{, sujeito {\`a}: }x=f(l,k)  \]
\[ \text{min }wl+rk  \text{, sujeito {\`a}: }y=g(l,k)  \]
O custo {\'e} minimizado quando: \[ TMST^x_{l,k} = \frac{w}{r} \text{   e   } TMST^y_{l,k} = \frac{w}{r} \]
Pode-se tamb{\'e}m agregar a demanda por capital e trabalho das empresas, somando-as horizontalmente: 
%FIGURA3
\section{Oferta das empresas: maximiza{\c c}{\~a}o dos lucros} 
Como assumimos competi{\c c}{\~a}o perfeita, temos que todos os produtores tem a mesma tecnologia e a curva de custo marginal {\'e} uma constante, dependendo de $w$ e $r$. 
Como a fun{\c c}{\~a}o de produ{\c c}{\~a}o de $x$ difere de $y$, as curva de $CM_x$ e $CM_y$ diferem tamb{\'e}m, dado que dependem de maneira diferente de $w$ e $r$. Como se
est{\'a} com concorr{\^e}ncia perfeita, os produtores agem como tomadores de pre{\c c}o e a oferta {\'e} igual ao custo marginal, al{\'e}m de ser igual ao pre{\c c}o de equil{\'i}brio(ainda n{\~a}o 
pode-se afirmar isso, mas ser{\'a} feito no decorrer do artigo). 
Como os retornos de escala s{\~a}o constantes, o custo m{\'e}dio e o custo marginal s{\~a}o iguais, o que garante lucro zero aos produtores. $CM_x$ e $CM_y$ dependem de
$w$ e $r$, que dependem da oferta e demanda do mercado de insumos, o que mostra que os mercaodos s{\~a}o interdependentes.
%FIGURA4
\section{Oferta das fam{\'i}lias: maximiza{\c c}{\~a}o dos seus lucros}
Trabalho e capital s{\~a}o ofertados pelas fam{\'i}lias, que ofertam uma quantidade fixa e s{\~a}o indiferentes a qual mercado ofertar desde que o pagamento($r$ ou $w$) seja 
igual nos dois mercados e maior que zero. Como a quantidade {\'e} fixa, as curvas de oferta $S_L$ e $S_K$ s{\~a}o retas verticais, lembrando que os oper{\'a}rios ofertam trabalho
e as fam{\'i}lias de escrit{\'o}rio ofertam capital. 
%FIGURA5
\section{An{\'a}lise de equil{\'i}brio geral}
Quando todas essas partes interagem, quatro pre{\c c}os ser{\~a}o determinados ao mesmo tempo: $P_x$, para a energia; $P_y$, para os alimentos; $w$ pelos servi{\c c}os do 
trabalho e $r$ pelos servi{\c c}os do capital, Todos s{\~a}o interdependentes: o pre{\c c}o da energia {\'e} determinado pelo custo marginal da energia, mas esse depende dos 
pre{\c c}os do trabalho e do capital. Os pre{\c c}os que equilibram os mercados s{\~a}o tais que: a demanda familiar de energia {\'e} igual {\`a} oferta de mercado de energia, 
a demanda familiar de alimentos {\'e} igual {\`a} oferta de mercado de alimentos, a demanda de mercado de trabalho {\'e} igual {\`a} oferta familiar de trabalho e a demanda de mercado
de capital {\'e} igual {\`a} oferta familiar de capital. Um oberva{\c c}{\~a}o a ser feita {\'e} que, como a quantidade de capital {\'e} menor, esse {\'e} melhor remunerado, deixando as 
fam{\'i}lias de trabalhadores de escrit{\'o}rio com maior renda que as fam{\'i}lias de oper{\'a}rios. 
%FIGURA6
\section{Lei de Walras}
Ao tentar resolver um sistema de N mercados, nota-se que se a oferta for igual a demanda em $N-1$ mercados, tal igual necessariamente ser{\'a} v{\'a}lida tamb{\'e}m 
no N-{\'e}simo mercado em um equil{\'i}brio geral comperitivo. Logo, em um sitema de equa{\c c}{\~o}es com tr{\^e}s condi{\c c}{\~o}es e quatro inc{\'o}gnitas, temos que uma delas n{\~a}o poder{\'a} ser 
descoberta, podendo se fixar o pre{\c c}o a qaulquer n{\'i}mero que se queira. Ent{\~a}o, a Lei diz que a an{\'a}lise de equil{\'i}brio geral determina os pre{\c c}os relativos, n{\~a}o
absolutos. No exemplo, fora fixado o pre{\c c}o do capital em 1 unidade e os outros pre{\c c}os(alimentos, energia e trabalho) foram determinados em rela{\c c}{\~a}o {\`a} ele. 
Este tipo de an{\'a}lise {\'e} usado pelos economistas para se analisar os efeitos de uma pol{\'i}tica p{\'u}blica na economia, como por exemplo uma introdu{\c c}{\~a}o de um imposto sobre
um certo bem, como a gasolina.



\end{document}